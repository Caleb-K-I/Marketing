% Options for packages loaded elsewhere
\PassOptionsToPackage{unicode}{hyperref}
\PassOptionsToPackage{hyphens}{url}
%
\documentclass[
  14pt,
  french,
]{article}
\usepackage{lmodern}
\usepackage{amsmath}
\usepackage{ifxetex,ifluatex}
\ifnum 0\ifxetex 1\fi\ifluatex 1\fi=0 % if pdftex
  \usepackage[T1]{fontenc}
  \usepackage[utf8]{inputenc}
  \usepackage{textcomp} % provide euro and other symbols
  \usepackage{amssymb}
\else % if luatex or xetex
  \usepackage{unicode-math}
  \defaultfontfeatures{Scale=MatchLowercase}
  \defaultfontfeatures[\rmfamily]{Ligatures=TeX,Scale=1}
\fi
% Use upquote if available, for straight quotes in verbatim environments
\IfFileExists{upquote.sty}{\usepackage{upquote}}{}
\IfFileExists{microtype.sty}{% use microtype if available
  \usepackage[]{microtype}
  \UseMicrotypeSet[protrusion]{basicmath} % disable protrusion for tt fonts
}{}
\makeatletter
\@ifundefined{KOMAClassName}{% if non-KOMA class
  \IfFileExists{parskip.sty}{%
    \usepackage{parskip}
  }{% else
    \setlength{\parindent}{0pt}
    \setlength{\parskip}{6pt plus 2pt minus 1pt}}
}{% if KOMA class
  \KOMAoptions{parskip=half}}
\makeatother
\usepackage{xcolor}
\IfFileExists{xurl.sty}{\usepackage{xurl}}{} % add URL line breaks if available
\IfFileExists{bookmark.sty}{\usepackage{bookmark}}{\usepackage{hyperref}}
\hypersetup{
  pdftitle={projet marketing},
  pdfauthor={KASHALA ILUNGA Caleb},
  pdflang={fr},
  hidelinks,
  pdfcreator={LaTeX via pandoc}}
\urlstyle{same} % disable monospaced font for URLs
\usepackage[a4paper,top=2cm,bottom=2cm,left=2cm,right=2cm]{geometry}
\usepackage{graphicx}
\makeatletter
\def\maxwidth{\ifdim\Gin@nat@width>\linewidth\linewidth\else\Gin@nat@width\fi}
\def\maxheight{\ifdim\Gin@nat@height>\textheight\textheight\else\Gin@nat@height\fi}
\makeatother
% Scale images if necessary, so that they will not overflow the page
% margins by default, and it is still possible to overwrite the defaults
% using explicit options in \includegraphics[width, height, ...]{}
\setkeys{Gin}{width=\maxwidth,height=\maxheight,keepaspectratio}
% Set default figure placement to htbp
\makeatletter
\def\fps@figure{htbp}
\makeatother
\usepackage[normalem]{ulem}
% Avoid problems with \sout in headers with hyperref
\pdfstringdefDisableCommands{\renewcommand{\sout}{}}
\setlength{\emergencystretch}{3em} % prevent overfull lines
\providecommand{\tightlist}{%
  \setlength{\itemsep}{0pt}\setlength{\parskip}{0pt}}
\setcounter{secnumdepth}{5}
\usepackage[Glenn]{fncychap}
\usepackage{fancyhdr}
\usepackage{booktabs}
\usepackage{longtable}
\usepackage{array}
\usepackage{multirow}
\usepackage{wrapfig}
\usepackage{float}
\usepackage{colortbl}
\usepackage{pdflscape}
\usepackage{tabu}
\usepackage{threeparttable}
\usepackage{threeparttablex}
\usepackage[normalem]{ulem}
\usepackage{makecell}
\usepackage{xcolor}
\ifxetex
  % Load polyglossia as late as possible: uses bidi with RTL langages (e.g. Hebrew, Arabic)
  \usepackage{polyglossia}
  \setmainlanguage[]{french}
\else
  \usepackage[shorthands=off,main=french]{babel}
\fi
\ifluatex
  \usepackage{selnolig}  % disable illegal ligatures
\fi

\title{projet marketing}
\author{KASHALA ILUNGA Caleb}
\date{03/04/2020}

\begin{document}
\maketitle

{
\setcounter{tocdepth}{2}
\tableofcontents
}
\newpage

\hypertarget{nos-variable-uxe9tudiuxe9es}{%
\section{Nos variable étudiées}\label{nos-variable-uxe9tudiuxe9es}}

-La base de données que nous allons étudier est une base de données qui
contient des informations sur diverses utilisations des services
médicaux. Les personnes âgées peuvent obtenir une assurance
complémentaire soit en l'achetant elles-mêmes ou en adhérant à des
régimes parrainés par l'employeur.Le but de l'étude que nous allons ici
réaliser,est de savoir si oui ou non les personnes âgées ont souscrit
une assurance complémentaire,la variable \emph{assuré} sera donc notre
variable à expliquer.Pour ce faire nous allons commencer par une études
globale sur la base de données puis ensuite nous allons procéder à une
estimation d'un modéle logit et probit et définir le meilleur modéle de
``prédiction''. \linebreak -Notre base de données contient 3206
observations et 20 variables. \linebreak Nous observons que notre base
de données a une varibale appelé ``X'' qui correspond à la numérotation
des lignes de chaque individu, nous l'avons donc retiré cette variable
car elle apporte aucune information importante dans l'étude que nous
aurons à mener ici. \linebreak Il nous reste donc 19 variables sur notre
base de données dont la variable à expliquer \emph{assuré} avec
lesquelles nous allons faire notre étude. - Nos variables explicatives
sont : \linebreak Pour toute les variables la valeur 1 veut dire OUI et
0 veut dire NON sauf celle dont nous allons le préciser.\\
\linebreak

\textbf{assurance\_privé} : Si oui ou non l'individu a souscrit une
assurance privée,prend la valeur 1 ou 0 .

\linebreak

\textbf{age} :Qui correspond à l'âge de l'individu.

\linebreak

\textbf{hispanique} :Si oui ou non l'individu est hispanique,prend la
valeur 1 et 0.

\linebreak

\textbf{blanc} :si oui ou non l'individu est blanc.

\linebreak

\textbf{femme} :Si oui ou non l'individu est une femmme,prend la valeur
1 et 0.

\linebreak

\textbf{année\_d\_éducation}: Qui correspond au nombre d'année d'étude
éffectuer par l'individus.

\linebreak

\textbf{marié}:Si oui ou non l'individu est marié.

\linebreak

\textbf{excellente\_santé}:Si la personne a oui ou non une excellente
santé,prend la valeur 1 et 0.

\linebreak

\textbf{trés\_bonne\_santé}:Si la personne a oui ou non une trés bonne
santé,prend la valeur 1 et 0.

\linebreak

\textbf{bonne\_santé}:Si la personne a oui ou non une bonne santé,prend
la valeur 1 et 0.

\linebreak

\textbf{santé\_passable}:Si la personne a oui ou non une santépassable
,prend la valeur 1 et 0.

\linebreak

\textbf{mauvaise\_santé}:Si la personne a oui ou non une mauvaise
santé,prend la valeur 1 et 0.

\linebreak

\textbf{maladie\_chronique}:Nombre de maladie chronique chez l'individu.

\linebreak

\textbf{adl}:le nombre de limitations (jusqu'à cinq) sur les activités
de la vie quotidienne.

\linebreak

\textbf{retraité}:Si l'individu est retaité.

\linebreak

\textbf{conjoint\_retraité}:Si le conjoint de l'individu est retraité.

\linebreak

\textbf{revenu}:revenu de l'individu.

\linebreak

\textbf{statut\_santé}:statut santé de l'individu.

\linebreak

\begin{itemize}
\tightlist
\item
  Notre variable à expliquer \emph{assuré}:
\end{itemize}

\includegraphics{Projet-Marketing-Caleb-KASHALA-ILUNGA_files/figure-latex/unnamed-chunk-1-1.pdf}
On constate que sur notre base de données il y'a environs 2000 indivus
qui ne sont pas souscrit à une assurance complémentaire et environs 1250
qui sont souscrit à une assurance complémentaire.

\includegraphics{Projet-Marketing-Caleb-KASHALA-ILUNGA_files/figure-latex/unnamed-chunk-2-1.pdf}
- On constate qu'il y'a bien plus de blanc que d'hispanique dans notre
base de données.

\includegraphics{Projet-Marketing-Caleb-KASHALA-ILUNGA_files/figure-latex/unnamed-chunk-3-1.pdf}
\includegraphics{Projet-Marketing-Caleb-KASHALA-ILUNGA_files/figure-latex/unnamed-chunk-3-2.pdf}
On contate que la majorité des individus ont des revenu reparti entre 0
et 250.Nous avons pas l'information pour savoir si les revenus sont
donnés dans quel ordre d'échelle. Nous avons mis une representaion du
revenu au logarithme pour montrer qu'il y'a un effet de lissage ce quin
permet de mieux capter les changements.\[\\\]
\includegraphics{Projet-Marketing-Caleb-KASHALA-ILUNGA_files/figure-latex/unnamed-chunk-4-1.pdf}

-À travers ce graphique on peut voir deux imformations.La premiere est
qu'il y'a plus d'homme que de femme dans notre base de données et la
deuxiéme et que chez les femmes comme chez les hommes il y' a plus
d'individu qu n'ont pas de d'assurance complémentaire.\[\\\]

\includegraphics{Projet-Marketing-Caleb-KASHALA-ILUNGA_files/figure-latex/unnamed-chunk-5-1.pdf}
- Comme on peut le voir sur ce graphique la tranche d'âge la plus
representé dans notre base de données et la tranche d'age 65 72 ans
environs.Et en majorité il y'a toujours plus de non souscrit à une
assurance complémentaire que des souscrit\[\\\]
\includegraphics{Projet-Marketing-Caleb-KASHALA-ILUNGA_files/figure-latex/unnamed-chunk-6-1.pdf}
-Ce graphique nous montre que la majorité des individus non souscrit et
souscrit ont effectué 12 ans d'étude.Et que la repartions d'individus
souscrit à l'assurance commence à partir de 8 ans études.\[\\\]
\includegraphics{Projet-Marketing-Caleb-KASHALA-ILUNGA_files/figure-latex/unnamed-chunk-7-1.pdf}
-on voit assez bien que la plus part d'individus souscrit ou pas à une
assurance complémentaire n'ont pas de restriction dans pour la limite
d'activités dans la vie quotidenne et que plus il y'a de restricton
moins la densité est grande.\[\\\]
\includegraphics{Projet-Marketing-Caleb-KASHALA-ILUNGA_files/figure-latex/unnamed-chunk-8-1.pdf}
-On remarque sur ce graphique que la majorité des personnes qui se
souscrivent à une assurance complémentaire on un bonne santé et que
c'est la même chose chez les non assuré.Donc on d'autre termes que la
santé n'a pas forcément une grande influence dans le fait d'avoir ou pas
une assurance complémentaire,il faudra le vérifier.\(\\\)

\hypertarget{moduxe9le-guxe9nuxe9ral-vraisemblance-et-log-vraisemblance.}{%
\section{Modéle général , Vraisemblance et
Log-vraisemblance.}\label{moduxe9le-guxe9nuxe9ral-vraisemblance-et-log-vraisemblance.}}

Le modéle probit et logit sont des modéle dichotomique par modéle
dichotomique, on entend un modéle statistique dans lequel la variable
expliquée ne peut prendre que deux modalités (variable dichotomique). Il
s'agit alors généralement d'expliquer la survenue ou non d'un événement,
ou d'un choix. On considére un échantillon de n individus d'indices i =
1, .., n.~ Pour chaque individu, on observe si un certain évènement
s'est réalisé et l'on pose:

\(Y_{i}\) =
\(\left\{\begin{array}{ll}1~ si~ l'evenement~ se ~ réalise~ ,Y^{*}\geq 0\\0~ sinon~ ,Y^{*}\leq 0\end{array}\right.\)\sout{avec}Y\^{}\{*\}=\(\beta X_{i}\)
+\(\varepsilon_{i}\)

\begin{itemize}
\tightlist
\item
  On utilise la méthode du maximum de vraisemblance pour estimer nos
  paramétres qui s'ecrit de façon suivante:
\end{itemize}

\(L(\theta)=\prod_{i=1}^{N}F( X_{i} \theta)^{Y_i}(1-F( X_{i} \theta))^{1-Y_i}\)

\begin{itemize}
\tightlist
\item
  Et sa log-vraisemblance qui s'ecrit de façon suivante
\end{itemize}

\(log(L(\theta))=\sum_{i:Y_i=1}logF( X_{i} \theta) + \sum_{i:Y_i=1}log(1-F( X_{i} \theta))\)

\begin{itemize}
\tightlist
\item
  Elle permet d'obtenir les différents coefficients associés à nos
  modéles .
\end{itemize}

\hypertarget{estimation-probit-et-logit}{%
\section{Estimation (probit et
logit)}\label{estimation-probit-et-logit}}

Nous allons ici vous présenter les Modéles probit et logit.

\hypertarget{probit}{%
\subsection{Probit}\label{probit}}

\begin{itemize}
\tightlist
\item
  Pour le modéle Probit on pause F ,qui est la fonction de répartition
  d'une gaussienne centrée réduite,usuellement notée Φ :\(\\\)
\end{itemize}

F( X\_\{i\} \theta)=\Phi( X\_\{i\}
\theta)=\int\emph{\{i=1\}\^{}\{X}\{i\}\theta\}
\frac{\exp{(-t²/2)}}{\sqrt{2\pi}}dt

\begin{itemize}
\tightlist
\item
  Et sa densité correspondante, usuellement notée φ, est :\(\\\) f(
  X\_\{i\} \theta)=\phi( X\_\{i\} \theta)=
  \frac{\exp{-((X_{i}  \theta)²/2)}}{\sqrt{2\pi}}dt \[\\\] \#\# Logit
\end{itemize}

-Pour le modéle Logit on pause F ,qui est la fonction de répartition
introduite spécialement pour ce type de modéle,usuellement notée
Λ:\(\\\)

F( X\_\{i\} \theta)=\Lambda( X\_\{i\} \theta)=
\frac{\exp{(X_{i}  \theta)}}{1+\exp(X_{i}  \theta)  }dt=\frac{1}{1+\exp(-X_{i}  \theta)}dt

\begin{itemize}
\tightlist
\item
  Et sa densité correspondante, usuellement notée λ, est :\[\\\]
\end{itemize}

f( X\_\{i\} \theta)=\lambda( X\_\{i\} \theta)=
\frac{\exp{(-X_{i}  \theta)}}{(1+\exp(-X_{i}  \theta))²  }dt=\Lambda(
X\_\{i\} \theta)(1-\Lambda( X\_\{i\} \theta))

\begin{itemize}
\tightlist
\item
  Il n'y a pratiquement pas de différence entre ces deux lois,
  l'introduction de la loi logistique étant simplement plus simple en
  terme de calcul en général. La seule différence notable entre les deux
  modéles probit et logit vient tout simplement de la spécification de
  la fonction de répartition F.
\end{itemize}

\hypertarget{hypothuxe9se-et-spuxe9cification-de-test}{%
\section{Hypothése et spécification de
test}\label{hypothuxe9se-et-spuxe9cification-de-test}}

\textbf{Test de wald}

Le test de Wald est un test paramétrique économétrique qui permet de
tester la ``vraie'' valeur du paramètre basé sur l'estimation de
l'échantillon.À savoir si nos paramétres sont siginificatifs ou pas.

le test s'écrit de façon suivante:

H0:\(\beta_k\) = 0 ,alors le paramétre est n'est pas significativement
différent de 0 \(\\\) H1:\(\beta_k\) \(\ne\) ,0 alors le paramétre est
significativement différent de 0 \(\\\)

\emph{la statistique:}

\[W= \frac{\beta²_k}{S²_k }\]

\emph{Décision}

On rejette H0 au risque \(\alpha\) si W \(\geq\) \(\chi²_{1-\alpha}(1)\)

\textbf{Test de spécification Hosmer-Lemeshow}

Le test de spécification Hosmer-Lemeshow est un test statistique pour la
qualité d'ajustement pour la régression logistique modèles, permet de
tester si les proportions observées et attendues diffèrent de manière
significative . soit G le nombre de groupe,les observations sont
regroupées par probabilité attendue. les observations avec une
probabilité attendue similaire sont regroupées dans le même groupe, pour
créer 10 groupes. soit p\_g la probabilité moyenne prédite dans le
groupe g soit y\_g la fréquence d'échantillonnage moyenne dans le groupe
g.\(\\\)

\emph{la statistique:}\(\\\)

\(HL=\sum^{G}_{g=1} \frac{(p_g-y_g)²}{y_g(1-y_g) }\)

Sous la valeur nulle de spécification correcte, la statistique est
distribuée comme \chi²(G-2).

\emph{Décision}\linebreak

Si la p-value est inférieur à l'alpha choisi alors l'hypothése nulle
selon laquelle les proportions observées et attendues sont les mêmes est
rejetée.\linebreak

\hypertarget{comparaison-des-moduxe9les}{%
\section{Comparaison des modéles}\label{comparaison-des-moduxe9les}}

\begin{itemize}
\tightlist
\item
  Nous avons éffectué une premiére régression probit et logit avec tout
  nos variables pour avoir une vue d'ensemble sur nos variables et
  données et pour essayer de voir la significativité de chaque varible
  et on s'aperçoit que toutes les variables sauf assurance privée et la
  ``constante'' n'ont pas de coéfficient \(\beta\) associé ce qui
  suggére une certaine corrélation forte entre l'un de nos variables
  explicatives et notre variable à expliquer.
\end{itemize}

\linebreak

le modéle est le suivant:

\linebreak

\begin{center}
$assuré_i$ = $\beta_0$ +  $\beta_1$ $assurance privé$  + $\beta_2$ $age$+ $\beta_3$ $hispanique$ + $\beta_4$ $blanc$ + $\beta_5$ $femme$  + $\beta_{6}$ $année d'éducation$ + $\beta_{7}$ $marié$ + $\beta_{8}$ $excellente santé$ +  $\beta_9$ $trés bonne santé$  + $\beta_10$ $bonne santé$+ $\beta_11$ $santé passable$ + $\beta_12$ $mauvaise santé$ + $\beta_13$ $maladie chronique$  + $\beta_{14}$ $adl$ + $\beta_{15}$ $retraité$ + $\beta_{16}$ $conjoint retraité$+ $\beta_{17}$ $revenu$ + $\beta_{18}$ $statut santé$ + $\varepsilon_i$ 
\end{center}

\begin{table}[!htbp] \centering 
  \caption{regression logit et probit avec  toute les variables} 
  \label{} 
\tiny 
\begin{tabular}{@{\extracolsep{5pt}}lcc} 
\\[-1.8ex]\hline 
\hline \\[-1.8ex] 
 & \multicolumn{2}{c}{\textit{Dependent variable:}} \\ 
\cline{2-3} 
\\[-1.8ex] & \multicolumn{2}{c}{assuré} \\ 
\\[-1.8ex] & \textit{logistic} & \textit{probit} \\ 
\\[-1.8ex] & (1) & (2)\\ 
\hline \\[-1.8ex] 
 assurance\_privé & 53.13213 & 13.98244 \\ 
  & (16,522.58000) & (3,432.36700) \\ 
  & & \\ 
 age & $-$0.00000 & 0.00000 \\ 
  & (2,306.54900) & (479.15820) \\ 
  & & \\ 
 hispanique & 0.00000 & 0.00000 \\ 
  & (32,311.46000) & (6,712.32800) \\ 
  & & \\ 
 blanc1 & $-$0.00000 & $-$0.00000 \\ 
  & (21,544.94000) & (4,475.70500) \\ 
  & & \\ 
 femme & 0.00000 & $-$0.00000 \\ 
  & (17,951.37000) & (3,729.18400) \\ 
  & & \\ 
 année\_d\_édu\_ation & $-$0.00000 & 0.00000 \\ 
  & (2,774.17800) & (576.30260) \\ 
  & & \\ 
 marié & $-$0.00000 & $-$0.00000 \\ 
  & (22,660.49000) & (4,707.44900) \\ 
  & & \\ 
 excellente\_santé & $-$0.00000 & $-$0.00000 \\ 
  & (42,117.78000) & (8,749.47100) \\ 
  & & \\ 
 trés\_bonne\_santé & $-$0.00000 & $-$0.00000 \\ 
  & (36,576.05000) & (7,598.24400) \\ 
  & & \\ 
 bonne\_santé & 0.00000 & 0.00000 \\ 
  & (34,416.78000) & (7,149.68200) \\ 
  & & \\ 
 santé\_passable & $-$0.00000 & $-$0.00000 \\ 
  & (33,561.38000) & (6,971.98200) \\ 
  & & \\ 
 mauvaise\_santé &  &  \\ 
  &  &  \\ 
  & & \\ 
 maladie\_chronique & 0.00000 & 0.00000 \\ 
  & (6,374.88200) & (1,324.30600) \\ 
  & & \\ 
 adl & 0.00000 & 0.00000 \\ 
  & (11,320.09000) & (2,351.61600) \\ 
  & & \\ 
 retraité & 0.00000 & 0.00000 \\ 
  & (17,451.86000) & (3,625.41600) \\ 
  & & \\ 
 conjoint\_retraité & $-$0.00000 & $-$0.00000 \\ 
  & (19,145.87000) & (3,977.32600) \\ 
  & & \\ 
 revenu & 0.00000 & 0.00000 \\ 
  & (123.85110) & (25.72858) \\ 
  & & \\ 
 statut\_santé &  &  \\ 
  &  &  \\ 
  & & \\ 
 Constant & $-$26.56607 & $-$6.99122 \\ 
  & (159,990.50000) & (33,236.13000) \\ 
  & & \\ 
\hline \\[-1.8ex] 
Observations & 2,138 & 2,138 \\ 
Log Likelihood & $-$0.00000 & $-$0.00000 \\ 
Akaike Inf. Crit. & 34.00000 & 34.00000 \\ 
\hline 
\hline \\[-1.8ex] 
\textit{Note:}  & \multicolumn{2}{r}{$^{*}$p$<$0.1; $^{**}$p$<$0.05; $^{***}$p$<$0.01} \\ 
\end{tabular} 
\end{table}

Et on s'aperçoit rapidement à travers ce graphique de matrice de
corrélation que ceci est dû à la forte corrélation que nous avons entre
la variable assurance privée et assuré.

\linebreak

De même nous constatons aussi des corrélation plus ou moins important
entre d'autres variables comme:

\linebreak

\begin{itemize}
\tightlist
\item
  statut stanté et bonne santé

  \begin{itemize}
  \tightlist
  \item
    statut stanté et mauvaise santé
  \item
    statut stanté et santé passable
  \item
    statut stanté et maladie chronique
  \end{itemize}
\end{itemize}

\includegraphics{Projet-Marketing-Caleb-KASHALA-ILUNGA_files/figure-latex/corrélation-1.pdf}

Pour la suite de notre étude, toute les estimations de nos modéles qui
seront réalisées sans la variable la variable \emph{``assurance
privée''}.

\linebreak

\hypertarget{mise-en-place-des-diffuxe9rent-moduxe9le}{%
\subsection{Mise en place des différent
modéle}\label{mise-en-place-des-diffuxe9rent-moduxe9le}}

\begin{itemize}
\tightlist
\item
  Le modéle 0 est le modéle suivant:
\end{itemize}

Dans ce modéle la variable ``statut santé'' a été retiré dû à sa
colinéarité de plus nous avons remarquer que cette même variable est une
variable qui traduit de façon global les informations liées aux cinq
variables associées à l'état de santé.

\begin{itemize}
\tightlist
\item
  le modéle 0 que nous proposant est donc le suivant:
\end{itemize}

\begin{center}
$assuré_i$ = $\beta_0$  + $\beta_1$ $age_i$+ $\beta_2$ $hispanique_i$ + $\beta_3$ $blanc_i$ + $\beta_4$ $femme_i$  + $\beta_{5}$ $année d'éducation_i$ + $\beta_{6}$ $marié_i$ + $\beta_{7}$ $excellente santé_i$ +  $\beta_8$ $trés bonne santé_i$  + $\beta_9$ $bonne santé_i$+ $\beta_10$ $santé passable_i$ + $\beta_11$ $mauvaise santé_i$ + $\beta_12$ $maladie chronique_i$  + $\beta_{13}$ $adl_i$ + $\beta_{14}$ $retraité_i$ + $\beta_{15}$ $conjoint retraité_i$+ $\beta_{16}$ $revenu_i$ + $\varepsilon_i$ 
\end{center}

\begin{itemize}
\item
  Pour l'éstimation de ce premier modéle nous avons enlever
  ``excellente\_santé'',``trés\_bonne\_santé'',``bonne\_santé'',``santé\_passable''
  et ``mauvaise santé'' dû à leur faible et moyenne corrélation avec la
  variable statut\_santé,dont nous avons également pu rémarquer qu'elle
  traduisait les informations de ces cinq variables liées à la santé.
\item
  Le modéle 1 est le modéle suivant:
\end{itemize}

\begin{center}
$assuré_i$ = $\beta_0$  + $\beta_1$ $age_i$+ $\beta_2$ $hispanique_i$ + $\beta_3$ $blanc_i$ + $\beta_4$ $femme_i$  + $\beta_{5}$ $année d'éducation_i$ + $\beta_{6}$ $marié_i$ + $\beta_7$ $maladie chronique_i$  + $\beta_{8}$ $adl_i$ + $\beta_{9}$ $retraité_i$ + $\beta_{10}$ $conjoint retraité_i$+ $\beta_{11}$ $revenu_i$+ $\beta_{12}$ $statut santé_i$ + $\varepsilon_i$ 
\end{center}

\begin{itemize}
\item
  Le modéle 0 et 1 nous permettra de jauger et savoir s'il est plus
  pertinent de garder les 5 variables associé à l'état de santé ou
  simplement de garder la variable ``statut sante'' et de retirer les
  cinq autres.
\item
  Pour l'éstimation de ce deuxiéme modéle nous avons éffectué un
  changement sur la variable age,que nous avons divisé en 3 classe
  d'age,une premiére classe de 52 à 64 ans une deuxiéme classe de 64 à
  76 ans et une troisiéme clase de 76 à 86 ans.
\item
  Le modéle 2 est le suivant:
\end{itemize}

\begin{center}
$assuré_i$ = $\beta_0$  + $\beta_1$ $age_i$+ $\beta_2$ $hispanique_i$ + $\beta_3$ $blanc_i$ + $\beta_4$ $femme_i$  + $\beta_{5}$ $année d'éducation_i$ + $\beta_{6}$ $marié_i$ + $\beta_7$ $maladie chronique_i$  + $\beta_{8}$ $adl_i$ + $\beta_{9}$ $retraité_i$ + $\beta_{10}$ $conjoint retraité_i$+ $\beta_{11}$ $revenu_i$+ $\beta_{12}$ $statut santé_i$ + $\varepsilon_i$ 
\end{center}

\begin{verbatim}
## [1] 52
\end{verbatim}

\begin{itemize}
\item
  Pour l'éstimation de ce troisiéme modéle nous avons appliquer le
  logarithme à la variable revenu pour baissé son effet d'echelle et
  ainsi mieux le comparer aux autres variables et ainsi pouvoir mieux
  capter les variations liéés à cette variable en la lissant.
\item
  Le modéle 3 est le modéle suivant:
\end{itemize}

\begin{center}
$assuré_i$ = $\beta_0$  + $\beta_1$ $age_i$+ $\beta_2$ $hispanique_i$ + $\beta_3$ $blanc_i$ + $\beta_4$ $femme_i$  + $\beta_{5}$ $année d'éducation_i$ + $\beta_{6}$ $marié_i$ + $\beta_7$ $maladie chronique_i$  + $\beta_{8}$ $adl_i$ + $\beta_{9}$ $retraité_i$ + $\beta_{10}$ $conjoint retraité_i$+ $\beta_{11}$ $lnrevenu_i$+ $\beta_{12}$ $statut santé_i$ + $\varepsilon_i$ 
\end{center}

Pour l'éstimation de ce quatriéme modele nous avons tous simplement mis
au carré la variable age dans le but d'éssayer de capter au mieux les
changement liés à cette variable.

\begin{itemize}
\tightlist
\item
  Le modéle 4 est le modéle suivant:
\end{itemize}

\begin{center}
$assuré_i$ = $\beta_0$  + $\beta_1$ $Age²_i$+ $\beta_2$ $hispanique_i$ + $\beta_3$ $blanc_i$ + $\beta_4$ $femme_i$  + $\beta_{5}$ $année d'éducation_i$ + $\beta_{6}$ $marié_i$ + $\beta_7$ $maladie chronique_i$  + $\beta_{8}$ $adl_i$ + $\beta_{9}$ $retraité_i$ + $\beta_{10}$ $conjoint retraité_i$+ $\beta_{11}$ $revenu_i$+ $\beta_{12}$ $statut santé_i$ + $\varepsilon_i$ 
\end{center}

Pour l'éstimation de ce cinquiéme modéle nous avons tout simplement
combiné l"age au carré et le logarithme sur le revenu .

\begin{itemize}
\tightlist
\item
  Le modéle 5 est le modéle suivant:
\end{itemize}

\begin{center}
$assuré_i$ = $\beta_0$  + $\beta_1$ $Age²_i$+ $\beta_2$ $hispanique_i$ + $\beta_3$ $blanc_i$ + $\beta_4$ $femme_i$  + $\beta_{5}$ $année d'éducation_i$ + $\beta_{6}$ $marié_i$ + $\beta_7$ $maladie chronique_i$  + $\beta_{8}$ $adl_i$ + $\beta_{9}$ $retraité_i$ + $\beta_{10}$ $conjoint retraité_i$+ $\beta_{11}$ $lnrevenu_i$+ $\beta_{12}$ $statut santé_i$ + $\varepsilon_i$ 
\end{center}

\begin{itemize}
\item
  Pour l'estimation de sixiéme modéle nous avons effectué un changement
  sur la variable statut\_santé,ainsi elle prend la valeur de:

  1 si l'individu a une excellente santé

  \linebreak

  2 si l'individu a une trés bonne santé

  \linebreak

  3 si l'individu a une bonne santé

  \linebreak

  4 si l'individu a une santé passable

  \linebreak

  5 si l'individu a une mauvaise santé

  \linebreak
\item
  Le modéle 6 est le modéle suivant:
\end{itemize}

\begin{center}
$assuré_i$ = $\beta_0$  + $\beta_1$ $Age²_i$+ $\beta_2$ $hispanique_i$ + $\beta_3$ $blanc_i$ + $\beta_4$ $femme_i$  + $\beta_{5}$ $année d'éducation_i$ + $\beta_{6}$ $marié_i$ + $\beta_7$ $maladie chronique_i$  + $\beta_{8}$ $adl_i$ + $\beta_{9}$ $retraité_i$ + $\beta_{10}$ $conjoint retraité_i$+ $\beta_{11}$ $lnrevenu_i$+ $\beta_{12}$ $statut santé_2$ + $\beta_{13}$ $statut santé_3$+ $\beta_{14}$ $statut santé_4$+ $\beta_{15}$ $statut santé_5$+ $\varepsilon_i$
\end{center}

\begin{itemize}
\item
  Pour l'estimation de septiéme modéle nous avons rajouté la variable
  age² et lnrevenu en plus des age et revenu que nous avions au départ.
\item
  Le modéle 7 est le modéle suivant:
\end{itemize}

\begin{center}
$assuré_i$ = $\beta_0$  + $\beta_1$ $age_i$+ $\beta_2$ $hispanique_i$ + $\beta_3$ $blanc_i$ + $\beta_4$ $femme_i$  + $\beta_{5}$ $année d'éducation_i$ + $\beta_{6}$ $marié_i$ + $\beta_7$ $maladie chronique_i$  + $\beta_{8}$ $adl_i$ + $\beta_{9}$ $retraité_i$ + $\beta_{10}$ $conjoint retraité_i$+ $\beta_{11}$ $revenu_i$+ $\beta_{12}$ $Age²_i$+ $\beta_{13}$ $lnrevenu_i$+ $\beta_{12}$ $statut santé_2$ + $\beta_{13}$ $statut santé_3$+ $\beta_{14}$ $statut santé_4$+ $\beta_{15}$ $statut santé_5$+ $\varepsilon_i$ 
\end{center}

\begin{itemize}
\tightlist
\item
  Aprés avoir mise en place nos sept modéles différent,nous allons
  sélectionner les modéles probits et logits qui présente les meilleurs
  critéres d'information et estimations sur les données
  d'entrainement.Ensuite nous allons essayer d'ameliorer nos différents
  modéles à travers plusieur procéder pour en tirer le meilleur modéle
  possible.
\end{itemize}

\linebreak

\begin{table}[!h]

\caption{\label{tab:AIC_BIC}Comparaison des AIC et BIC modèles}
\centering
\fontsize{12}{14}\selectfont
\begin{tabular}[t]{l|c|c}
\hline
  & AIC & BIC\\
\hline
\cellcolor{gray!6}{Modéle logit 0} & \cellcolor{gray!6}{2685.494} & \cellcolor{gray!6}{2776.176}\\
\hline
Modéle probit 0 & 2684.614 & 2775.296\\
\hline
\cellcolor{gray!6}{Modéle logit 1} & \cellcolor{gray!6}{2683.492} & \cellcolor{gray!6}{2757.172}\\
\hline
Modéle probit 1 & 2682.314 & 2755.993\\
\hline
\cellcolor{gray!6}{Modéle logit 2} & \cellcolor{gray!6}{2684.144} & \cellcolor{gray!6}{2780.478}\\
\hline
Modéle probit 2 & 2683.200 & 2779.534\\
\hline
\cellcolor{gray!6}{Modéle logit 3} & \cellcolor{gray!6}{2617.711} & \cellcolor{gray!6}{2680.055}\\
\hline
Modéle probit 3 & 2614.359 & 2676.703\\
\hline
\cellcolor{gray!6}{Modéle logit 4} & \cellcolor{gray!6}{2683.627} & \cellcolor{gray!6}{2751.639}\\
\hline
Modéle probit 4 & 2682.368 & 2750.380\\
\hline
\cellcolor{gray!6}{Modéle logit 5} & \cellcolor{gray!6}{2617.647} & \cellcolor{gray!6}{2685.658}\\
\hline
Modéle probit 5 & 2614.251 & 2682.262\\
\hline
\cellcolor{gray!6}{Modéle logit 6} & \cellcolor{gray!6}{2685.494} & \cellcolor{gray!6}{2776.176}\\
\hline
Modéle probit 6 & 2684.614 & 2775.296\\
\hline
\cellcolor{gray!6}{Modéle logit 7} & \cellcolor{gray!6}{2600.871} & \cellcolor{gray!6}{2702.888}\\
\hline
Modéle probit 7 & 2598.469 & 2700.487\\
\hline
\end{tabular}
\end{table}

\begin{itemize}
\tightlist
\item
  Aprés avoir pu étudier le critère d'information d'Akaike (AIC,Akaike
  information criterion) qui est une mesure de la qualité d'un modèle
  statistique ainsi que le critère d'information bayésien (BIC,bayesian
  information criterion) dérivé du critère d'information d'Akaike.Nous
  avons sélectionné les modéles qui présente les plus faibles mesure
  d'AIC et de BIC combinés. Les modéles 5 et 7 sont les modéles que nous
  avons séléctionné pour la suite de l'étude que nous allons mener. À
  travers ces deux modéles nous allons comparer les modéles probit et
  logit. -Notre modéle 5 est le suivant
\end{itemize}

\begin{center}
$assuré_i$ = $\beta_0$  + $\beta_1$ $Age²_i$+ $\beta_2$ $hispanique_i$ + $\beta_3$ $blanc_i$ + $\beta_4$ $femme_i$  + $\beta_{5}$ $année d'éducation_i$ + $\beta_{6}$ $marié_i$ + $\beta_7$ $maladie chronique_i$  + $\beta_{8}$ $adl_i$ + $\beta_{9}$ $retraité_i$ + $\beta_{10}$ $conjoint retraité_i$+ $\beta_{11}$ $lnrevenu_i$+ $\beta_{12}$ $statut santé_i$ + $\varepsilon_i$ 
\end{center}

-Notre modéle 7 est le suivant:

\begin{center}
$assuré_i$ = $\beta_0$  + $\beta_1$ $age_i$+ $\beta_2$ $hispanique_i$ + $\beta_3$ $blanc_i$ + $\beta_4$ $femme_i$  + $\beta_{5}$ $année d'éducation_i$ + $\beta_{6}$ $marié_i$ + $\beta_7$ $maladie chronique_i$  + $\beta_{8}$ $adl_i$ + $\beta_{9}$ $retraité_i$ + $\beta_{10}$ $conjoint retraité_i$+ $\beta_{11}$ $revenu_i$+ $\beta_{12}$ $Age²_i$+ $\beta_{13}$ $lnrevenu_i$+ $\beta_{12}$ $statut santé_2$ + $\beta_{13}$ $statut santé_3$+ $\beta_{14}$ $statut santé_4$+ $\beta_{15}$ $statut santé_5$+ $\varepsilon_i$ 
\end{center}

\begin{table}[!htbp] \centering 
  \caption{regression logit et probit modéle 5} 
  \label{} 
\tiny 
\begin{tabular}{@{\extracolsep{5pt}}lcc} 
\\[-1.8ex]\hline 
\hline \\[-1.8ex] 
 & \multicolumn{2}{c}{\textit{Dependent variable:}} \\ 
\cline{2-3} 
\\[-1.8ex] & \multicolumn{2}{c}{assuré} \\ 
\\[-1.8ex] & \textit{logistic} & \textit{probit} \\ 
\\[-1.8ex] & (1) & (2)\\ 
\hline \\[-1.8ex] 
 Age² & $-$0.00010 & $-$0.00007 \\ 
  & (0.00011) & (0.00006) \\ 
  & & \\ 
 hispanique & $-$0.72104$^{***}$ & $-$0.42336$^{***}$ \\ 
  & (0.25360) & (0.14262) \\ 
  & & \\ 
 blanc & $-$0.18087 & $-$0.10826 \\ 
  & (0.13610) & (0.08212) \\ 
  & & \\ 
 femme & $-$0.02093 & $-$0.01639 \\ 
  & (0.10910) & (0.06653) \\ 
  & & \\ 
 année\_d\_édu\_ation & 0.09560$^{***}$ & 0.05872$^{***}$ \\ 
  & (0.01847) & (0.01106) \\ 
  & & \\ 
 marié & 0.09670 & 0.06192 \\ 
  & (0.15041) & (0.09087) \\ 
  & & \\ 
 retraité & 0.15061 & 0.09228 \\ 
  & (0.10684) & (0.06489) \\ 
  & & \\ 
 conjoint\_retraité & 0.01514 & 0.00650 \\ 
  & (0.11474) & (0.07021) \\ 
  & & \\ 
 lnrevenu & 0.62068$^{***}$ & 0.37752$^{***}$ \\ 
  & (0.07495) & (0.04462) \\ 
  & & \\ 
 statut\_santé & $-$0.02650 & $-$0.01050 \\ 
  & (0.12408) & (0.07505) \\ 
  & & \\ 
 adl & $-$0.13845$^{*}$ & $-$0.08416$^{*}$ \\ 
  & (0.07572) & (0.04428) \\ 
  & & \\ 
 Constant & $-$3.22173$^{***}$ & $-$1.96702$^{***}$ \\ 
  & (0.54526) & (0.32671) \\ 
  & & \\ 
\hline \\[-1.8ex] 
Observations & 2,138 & 2,138 \\ 
Log Likelihood & $-$1,296.82300 & $-$1,295.12500 \\ 
Akaike Inf. Crit. & 2,617.64700 & 2,614.25100 \\ 
\hline 
\hline \\[-1.8ex] 
\textit{Note:}  & \multicolumn{2}{r}{$^{*}$p$<$0.1; $^{**}$p$<$0.05; $^{***}$p$<$0.01} \\ 
\end{tabular} 
\end{table}

\begin{table}[!htbp] \centering 
  \caption{regression logit et probit modéle 7} 
  \label{} 
\tiny 
\begin{tabular}{@{\extracolsep{5pt}}lcc} 
\\[-1.8ex]\hline 
\hline \\[-1.8ex] 
 & \multicolumn{2}{c}{\textit{Dependent variable:}} \\ 
\cline{2-3} 
\\[-1.8ex] & \multicolumn{2}{c}{assuré} \\ 
\\[-1.8ex] & \textit{logistic} & \textit{probit} \\ 
\\[-1.8ex] & (1) & (2)\\ 
\hline \\[-1.8ex] 
 age & 0.20674 & 0.12900 \\ 
  & (0.25113) & (0.14949) \\ 
  & & \\ 
 blanc & $-$0.21510 & $-$0.13011 \\ 
  & (0.13863) & (0.08326) \\ 
  & & \\ 
 hispanique & $-$0.65206$^{**}$ & $-$0.38452$^{***}$ \\ 
  & (0.25638) & (0.14470) \\ 
  & & \\ 
 femme & $-$0.03859 & $-$0.03085 \\ 
  & (0.11025) & (0.06714) \\ 
  & & \\ 
 année\_d\_édu\_ation & 0.09531$^{***}$ & 0.05908$^{***}$ \\ 
  & (0.01881) & (0.01125) \\ 
  & & \\ 
 marié & $-$0.01276 & $-$0.00259 \\ 
  & (0.15447) & (0.09325) \\ 
  & & \\ 
 maladie\_chronique & 0.08897$^{**}$ & 0.05342$^{**}$ \\ 
  & (0.04022) & (0.02431) \\ 
  & & \\ 
 adl & $-$0.12234 & $-$0.07535 \\ 
  & (0.08002) & (0.04691) \\ 
  & & \\ 
 retraité & 0.10647 & 0.06558 \\ 
  & (0.10815) & (0.06566) \\ 
  & & \\ 
 conjoint\_retraité & 0.00092 & 0.00027 \\ 
  & (0.11557) & (0.07065) \\ 
  & & \\ 
 revenu & $-$0.00499$^{***}$ & $-$0.00275$^{***}$ \\ 
  & (0.00124) & (0.00069) \\ 
  & & \\ 
 statut\_santé2 & 0.19908 & 0.11887 \\ 
  & (0.16316) & (0.10002) \\ 
  & & \\ 
 statut\_santé3 & 0.11026 & 0.06950 \\ 
  & (0.16884) & (0.10317) \\ 
  & & \\ 
 statut\_santé4 & 0.17574 & 0.10032 \\ 
  & (0.20041) & (0.12170) \\ 
  & & \\ 
 statut\_santé5 & $-$0.09505 & $-$0.05467 \\ 
  & (0.27723) & (0.16571) \\ 
  & & \\ 
 age² & $-$0.00162 & $-$0.00101 \\ 
  & (0.00185) & (0.00110) \\ 
  & & \\ 
 lnrevenu & 0.96857$^{***}$ & 0.56906$^{***}$ \\ 
  & (0.11395) & (0.06559) \\ 
  & & \\ 
 Constant & $-$11.40848 & $-$6.98772 \\ 
  & (8.51336) & (5.06577) \\ 
  & & \\ 
\hline \\[-1.8ex] 
Observations & 2,138 & 2,138 \\ 
Log Likelihood & $-$1,282.43500 & $-$1,281.23500 \\ 
Akaike Inf. Crit. & 2,600.87000 & 2,598.46900 \\ 
\hline 
\hline \\[-1.8ex] 
\textit{Note:}  & \multicolumn{2}{r}{$^{*}$p$<$0.1; $^{**}$p$<$0.05; $^{***}$p$<$0.01} \\ 
\end{tabular} 
\end{table}

La différence que nous remarquons en premier entre les deux modéles est
que les estimation \(\beta\) associées aux variables explicatives du
modéle probit sont toujours plus faibles que les estimation \(\beta\)
associées aux variables explicatives du modéle logit.Cette différence
découle probablement de la différence de leur fonction de répartition F
et de densité f.

De plus nous avons aussi remarqué que les AIC et BIC du modéle probit
sont toujours un peu plus faible que celle du logit en général.

\begin{itemize}
\tightlist
\item
  Les deux tableau qui vont suivre nous donnes la valeurs des odds
  ration associés à nosvariables explicatives ainsi que leurs
  intervalles de confiance qui peut nous renseigner en amont surla
  significativité d'un coefficient.Un peut plus tard dans l'etude nous
  expliquerons le rôle des odds ration et qu'elle imformations nous
  pouvons en tirer de ces odds ratio qui sont généralement compris entre
  0 et +\(\infty\)
\end{itemize}

\textbackslash begin\{center\} \textbf{Tableau des odds ratio et
intervalles de confiance pour le modéle logit 5}

\includegraphics{Projet-Marketing-Caleb-KASHALA-ILUNGA_files/figure-latex/FOREST5-1.pdf}
\textbackslash end\{center\}

\textbackslash begin\{center\} \textbf{Tableau des odds ratio et
intervalles de confiance pour le modéle logit 7}
\includegraphics{Projet-Marketing-Caleb-KASHALA-ILUNGA_files/figure-latex/FOREST7-1.pdf}
\textbackslash end\{center\}

\hypertarget{pruxe9diction-et-fitted}{%
\section{Prédiction et fitted}\label{pruxe9diction-et-fitted}}

\begin{itemize}
\tightlist
\item
  Pour la suite de l'étude ,aprés avoir sélectionner nos modéles 3 et 5
  nous avons effectués une selection du meilleur modéle à partir de ces
  deux modéle de départ à travers une sélection par AIC pour déterminer
  le modéle 5 et 7 ``finaux'' puis nous allons les comparer.
\end{itemize}

\linebreak

Le modéle 5 ``final'' est le suivant:

\begin{center}
$assuré_i$ = $\beta_0$  + $\beta_1$ $hispanique_i$ + $\beta_{2}$ $année d'éducation_i$ + $\beta_{3}$ $adl_i$ + $\beta_{4}$ $lnrevenu_i$+ $\varepsilon_i$ 
\end{center}

Le modéle 7 ``final'' est le suivant:

\begin{center}
$assuré_i$ = $\beta_0$  + $\beta_1$ $hispanique_i$ + $\beta_2$ $blanc_i$  + $\beta_{3}$ $année d'éducation_i$ + $\beta_4$ $maladie chronique_i$  + $\beta_{5}$ $adl_i$ + $\beta_{6}$ $revenu_i$+ $\beta_{7}$ $lnrevenu_i$+ $\varepsilon_i$ 
\end{center}

\begin{itemize}
\tightlist
\item
  Ces graphique ci-dessous nous montre visuellement la relation qu'à
  notre variables à expliquer ``assuré'' avec nos différent variables
  explicatives,à savoir si elle ont réellement une influence sur
  celle-ci.
\end{itemize}

\newpage

\textbf{GRAPHIQUE SUR LES LIENS ENTRE LA VARIABLE À EXPLIQUER ET LES
VARIABLES EXPLICATIVES}

\includegraphics{Projet-Marketing-Caleb-KASHALA-ILUNGA_files/figure-latex/FITED5.1-1.pdf}

\includegraphics{Projet-Marketing-Caleb-KASHALA-ILUNGA_files/figure-latex/FITTED7-1.pdf}

\hypertarget{training}{%
\subsection{training}\label{training}}

\textbf{TABLEAU DE CONTINGENCE SUR LES DONNÉE D'APPRENTISSAGE}

\begin{center}

\begin{tabular}{l|r|r}
\hline
\multicolumn{1}{c|}{ } & \multicolumn{2}{c}{Modéle logit 5 } \\
\cline{2-3}
  & 0 & 1\\
\hline
0 & 1023 & 502\\
\hline
1 & 268 & 345\\
\hline
\end{tabular}


\begin{tabular}{l|r|r}
\hline
\multicolumn{1}{c|}{ } & \multicolumn{2}{c}{Modéle logit 5 final } \\
\cline{2-3}
  & 0 & 1\\
\hline
0 & 1025 & 514\\
\hline
1 & 266 & 333\\
\hline
\end{tabular}


\begin{tabular}{l|r|r}
\hline
\multicolumn{1}{c|}{ } & \multicolumn{2}{c}{Modéle probit 5 } \\
\cline{2-3}
  & 0 & 1\\
\hline
0 & 1025 & 508\\
\hline
1 & 266 & 339\\
\hline
\end{tabular}


\begin{tabular}{l|r|r}
\hline
\multicolumn{1}{c|}{ } & \multicolumn{2}{c}{Modéle probit 5 final } \\
\cline{2-3}
  & 0 & 1\\
\hline
0 & 1027 & 518\\
\hline
1 & 264 & 329\\
\hline
\end{tabular}


\begin{tabular}{l|r|r}
\hline
\multicolumn{1}{c|}{ } & \multicolumn{2}{c}{Modéle logit 7 } \\
\cline{2-3}
  & 0 & 1\\
\hline
0 & 994 & 452\\
\hline
1 & 297 & 395\\
\hline
\end{tabular}


\begin{tabular}{l|r|r}
\hline
\multicolumn{1}{c|}{ } & \multicolumn{2}{c}{Modéle logit 7 final } \\
\cline{2-3}
  & 0 & 1\\
\hline
0 & 998 & 453\\
\hline
1 & 293 & 394\\
\hline
\end{tabular}


\begin{tabular}{l|r|r}
\hline
\multicolumn{1}{c|}{ } & \multicolumn{2}{c}{Modéle probit 7 } \\
\cline{2-3}
  & 0 & 1\\
\hline
0 & 1000 & 456\\
\hline
1 & 291 & 391\\
\hline
\end{tabular}


\begin{tabular}{l|r|r}
\hline
\multicolumn{1}{c|}{ } & \multicolumn{2}{c}{Modéle probit 7 final} \\
\cline{2-3}
  & 0 & 1\\
\hline
0 & 1004 & 458\\
\hline
1 & 287 & 389\\
\hline
\end{tabular}


\begin{table}[!h]

\caption{\label{tab:TRREURA}Taux d'erreur sur les données d'apprentissage}
\centering
\fontsize{10}{12}\selectfont
\begin{tabular}[t]{l|c}
\hline
  & Taux d'érreur\\
\hline
\cellcolor{gray!6}{Modéle logit 5} & \cellcolor{gray!6}{0.3503274}\\
\hline
Modéle logit 5 final & 0.3489242\\
\hline
\cellcolor{gray!6}{Modéle probit 5} & \cellcolor{gray!6}{0.3493920}\\
\hline
Modéle probit 5 final & 0.3484565\\
\hline
\cellcolor{gray!6}{Modéle logit 7} & \cellcolor{gray!6}{0.3601497}\\
\hline
Modéle logit 7 final & 0.3648269\\
\hline
\cellcolor{gray!6}{Modéle probit 7} & \cellcolor{gray!6}{0.3620206}\\
\hline
Modéle probit 7 final & 0.3657624\\
\hline
\end{tabular}
\end{table}
\end{center}

\hypertarget{test}{%
\subsection{TEST}\label{test}}

\textbf{TABLEAU DE CONTINGENCE SUR LES DONNÉE TEST}

\begin{center}

\begin{tabular}{l|r|r}
\hline
\multicolumn{1}{c|}{ } & \multicolumn{2}{c}{Modéle logit 5 } \\
\cline{2-3}
  & 0 & 1\\
\hline
0 & 541 & 255\\
\hline
1 & 133 & 139\\
\hline
\end{tabular}


\begin{tabular}{l|r|r}
\hline
\multicolumn{1}{c|}{ } & \multicolumn{2}{c}{Modéle logit 5 final } \\
\cline{2-3}
  & 0 & 1\\
\hline
0 & 540 & 254\\
\hline
1 & 134 & 140\\
\hline
\end{tabular}


\begin{tabular}{l|r|r}
\hline
\multicolumn{1}{c|}{ } & \multicolumn{2}{c}{Modéle probit 5 } \\
\cline{2-3}
  & 0 & 1\\
\hline
0 & 542 & 258\\
\hline
1 & 132 & 136\\
\hline
\end{tabular}


\begin{tabular}{l|r|r}
\hline
\multicolumn{1}{c|}{ } & \multicolumn{2}{c}{Modéle probit 5 final } \\
\cline{2-3}
  & 0 & 1\\
\hline
0 & 543 & 255\\
\hline
1 & 131 & 139\\
\hline
\end{tabular}


\begin{tabular}{l|r|r}
\hline
\multicolumn{1}{c|}{ } & \multicolumn{2}{c}{Modéle logit 7 } \\
\cline{2-3}
  & 0 & 1\\
\hline
0 & 523 & 227\\
\hline
1 & 151 & 167\\
\hline
\end{tabular}


\begin{tabular}{l|r|r}
\hline
\multicolumn{1}{c|}{ } & \multicolumn{2}{c}{Modéle logit 7 final } \\
\cline{2-3}
  & 0 & 1\\
\hline
0 & 532 & 225\\
\hline
1 & 142 & 169\\
\hline
\end{tabular}


\begin{tabular}{l|r|r}
\hline
\multicolumn{1}{c|}{ } & \multicolumn{2}{c}{Modéle probit 7} \\
\cline{2-3}
  & 0 & 1\\
\hline
0 & 526 & 227\\
\hline
1 & 148 & 167\\
\hline
\end{tabular}


\begin{tabular}{l|r|r}
\hline
\multicolumn{1}{c|}{ } & \multicolumn{2}{c}{Modéle probit 7 final} \\
\cline{2-3}
  & 0 & 1\\
\hline
0 & 534 & 229\\
\hline
1 & 140 & 165\\
\hline
\end{tabular}




\begin{table}[!h]

\caption{\label{tab:TERREURT}Taux d'erreur sur les données test}
\centering
\fontsize{10}{12}\selectfont
\begin{tabular}[t]{l|c}
\hline
  & Taux d'érreur\\
\hline
\cellcolor{gray!6}{Modéle logit 5} & \cellcolor{gray!6}{0.3632959}\\
\hline
Modéle logit 5 final & 0.3632959\\
\hline
\cellcolor{gray!6}{Modéle probit 5} & \cellcolor{gray!6}{0.3651685}\\
\hline
Modéle probit 5 final & 0.3614232\\
\hline
\cellcolor{gray!6}{Modéle logit 7} & \cellcolor{gray!6}{0.3539326}\\
\hline
Modéle logit 7 final & 0.3436330\\
\hline
\cellcolor{gray!6}{Modéle probit 7} & \cellcolor{gray!6}{0.3511236}\\
\hline
Modéle probit 7 final & 0.3455056\\
\hline
\end{tabular}
\end{table}
\end{center}

D'aprés le calcul des taux d'erreur les estimation du modéle logit 7
``final'' et les estimations du modéle probit 7 ``final'' semble être
les meilleurs tout en notant que l'erreur sur le modéle logit 7 final
est faiblement plus petite comparer à l'erreur sur le modéle 7 probit
final.Pour la suite nous nous focaliserons sur le modéle logit 7 final.

\hypertarget{odds-ratio-test-et-effet-marginal.}{%
\section{Odds ratio ,test et effet
marginal.}\label{odds-ratio-test-et-effet-marginal.}}

\hypertarget{odds-ratio}{%
\subsection{ODDS-RATIO}\label{odds-ratio}}

\begin{itemize}
\tightlist
\item
  Bien que l'odds-ratio est d'une valeur qui ne s'interprete pas
  aisément néanmoins elle donne quelques informations globales.L'odds
  ratio s'interprète comme le risque relatif à la réalisation d'un
  évenement.Si l'odds ration est inférieur à un alors on a moins de
  ``chance'' à la survenue d'un évenement par rapport à une autre.Si
  l'odds ration est supérieur à un alors on a plus de ``chance'' à la
  survenue d'un évenement par rapport à une autre.Si elle est egale à
  zéro alors les ``chances'' des deux évenement sont les même.Pour faire
  simple dans notre cas l'odds ratio nous montre si la variable
  explicative à un effet positif (de chance) ou négatif(malchance) sur
  le fait d'être assuré.
\end{itemize}

\includegraphics{Projet-Marketing-Caleb-KASHALA-ILUNGA_files/figure-latex/REGL7-1.pdf}
- Ce tableau nous donne dans un premier temps la valeur de l'odds ratio
associée à chaque variables explcatives.On peut aussi voir l'intervalle
de confiance associé à chaque odds ration,ce qui peut nous donner une
information sur la significatvité de l'odds ratio.

-D'aprés ce tableau ,on peut déja observer que les odds ratio associés
aux variables ``blanc'',``hispanique'',``adl'' et ``revenu'' sont
inférieur à 1,ce qui nous suggére que ces variables là ont un effet
dit``négatif'' sur le fait d'être assuré. Et que les variables ``année
d'éducation'', ``maladie chronique'' et ``lnrevenue'' ont un effet
positif sur l'évenement être assuré. De plus on remarque la valeur assez
élevée du odds ration lié à la variable ``lnrevenu''

\hypertarget{test-1}{%
\subsection{Test}\label{test-1}}

\hypertarget{test-de-wald}{%
\subsubsection{Test de wald}\label{test-de-wald}}

\begin{verbatim}
## Wald test:
## ----------
## 
## Chi-squared test:
## X2 = 68.694, df = 6, P(> X2) = 7.5728e-13
\end{verbatim}

Pour notre test on rejette l'hypothèse H0 si la p-value est inférieure à
5\%. \linebreak Ici notre p.value vaut 7.5728e-13 donc au rique de 5\%
on rejette HO et on accepte l'hypothése que nos coéfficient sont
différent de 0.

\hypertarget{test-de-spuxe9cification-hosmer-lemeshow}{%
\subsubsection{Test de spécification
Hosmer-Lemeshow}\label{test-de-spuxe9cification-hosmer-lemeshow}}

\begin{verbatim}
## 
##  Hosmer and Lemeshow goodness of fit (GOF) test
## 
## data:  regl7_l$data$assuré, fitted(regl7_l)
## X-squared = 28.976, df = 8, p-value = 0.0003202
\end{verbatim}

\begin{verbatim}
## 
##  Hosmer and Lemeshow test (binary model)
## 
## data:  regl7_l$data$assuré, fitted(regl7_l)
## X-squared = 28.976, df = 8, p-value = 0.0003202
\end{verbatim}

Notre modéle à une p-value inférieur à 5\% ce qui suggére que nous avons
une erreur d'étalonnage globale au risque de 5\%.

\hypertarget{effet-marginal}{%
\subsection{Effet marginal}\label{effet-marginal}}

\begin{verbatim}
## Call:
## logitmfx(formula = regl7_l, data = test_data7, atmean = TRUE)
## 
## Marginal Effects:
##                         dF/dx   Std. Err.       z     P>|z|    
## blanc              0.01994744  0.04161988  0.4793   0.63174    
## hispanique        -0.10991305  0.05913847 -1.8586   0.06309 .  
## année_d_édu_ation  0.00261051  0.00564049  0.4628   0.64350    
## maladie_chronique  0.00999343  0.01142885  0.8744   0.38190    
## adl               -0.04253592  0.02228103 -1.9091   0.05625 .  
## revenu            -0.00338054  0.00072055 -4.6916 2.711e-06 ***
## lnrevenu           0.34308394  0.04210969  8.1474 3.719e-16 ***
## ---
## Signif. codes:  0 '***' 0.001 '**' 0.01 '*' 0.05 '.' 0.1 ' ' 1
## 
## dF/dx is for discrete change for the following variables:
## 
## [1] "blanc"      "hispanique"
\end{verbatim}

-L'effet marginal de la variable blanc est de 0.02 , cela equivaut à
dire que si un individu est blanc alors la probabilité qu'il ait une
assurance complémentaire augmente de 2 point. -L'effet marginal
correspondant à la variable hispanique vaut -0.11,ceci nous dit que si
une personne est d'hispanique alors la probabilité qu'il ait une
assurance complémentaire baisse de 10 point.

-L'effet marginal associé à la variable année d'éducation vaut 0.002,on
peut donc dire qu'en moyenneune variation positif de 1\% sur l'année
d'étude augmente la probababilité d'être souscrit à une complémentaire
santé augmente de 0.2 point.

-L'effet marginal associé à la variable maladie chronique vaut 0.010,on
peut donc dire qu'en moyenne une variation positif de 1\% sur la
variable maladie chronique augmente la probabilité d'avoir une
souscrition à une assurance complémentaire de 1 point.

-L'effet marginal correspondant à la variable adl vaut -0.04 ce qui
indique qu'en moyenne une variation positif de 1\% sur la variable adl
augmente la probabilité d'avoir une souscription à une assurance
complémentaire de 4 point.

-L'effet marginal correspondant à la variable revenu vaut -0.003 ce qui
nous emméne à dire qu'en moyenne une variation positif de 1\% sur le
revenu baisse la probabilité d'avoir une souscription à l'assurance
complémentaire de 0.3 point.

-L'effet marginal correspondant à notre variable lnrevenu vaut 0.34,ce
qui veut dire qu'en une variation positif de 1\% sur le revenu au
logarithme augmente la probabilité d'avoir une souscription à
l'assurance complémentaire de 34 point.

\hypertarget{interpruxe9tation}{%
\section{Interprétation}\label{interpruxe9tation}}

\begin{itemize}
\tightlist
\item
  On peut interpreter l'effet marginaux positf sur les blancs et négatif
  sur les hispaniques en disant que l'étude a été faite aux états-unis
  ,où il y'a une différence entre les blancs et les hispaniques qui
  viennent de l'amérique du sud genéralement clandestinements .Les
  blancs trés majoritaire occupent des bons postes tandis que les
  immigrés hispaniques ont des postes moins bien remunérés donc n'ont
  pas forcément le moyen de se payer une souscription à une assurance
  complémentaire contrairement aux indivudus blancs qui dans la moyenne
  le peuvent.
\end{itemize}

-Les variables année d'éducation et maladie chronique ont tout les deux
des éffets marginaux positifs qui peut s'expliquer par le fait que plus
on étudie plus on a un revenu et donc se permettre une complémentaire
santé.Et généralement quand on est sujet à revenr continuellement dans
un hopital comme quand on a une maladie chronique on opte généralement
pour une complémentaire santé.

-Pour les variables revenu et lnrevenu on a effet marginaux qui
s'oppose,on va tenter de l'expliquer. D'abord il est nécécessaire de
dire que la majorité des individus avait des revenu pas trés élevé ,donc
de ce fait on peut dire que n'ayant pas forcément des revenus illimités
les personnes à faible revenu préferent tout simplement se souscrire à
une complémentaire santé et ainsi en cas de problémes médicaux il ne
débourseront pas leurs argents.Alors qu'une personne aisé elle n'a pas
forcement ce besoin là elle peut elle même se payer ses séjours à
l'hôpital et ne pas opter pour une assurance complémentaire .Donc si le
revenu augmente ils vont de moins en moins souscrire une assurance
complémentaire.

\linebreak

Et d'un autre coté on peut se dire que justement une personne qui n'est
pas assez riche et/ou qui as souvent des problémes médicaux va préférer
garder son argent pour ses loisirs,son alimentation ou autre probléme
alors qu'une personne moyennement riche ou riche va elle se prevenir de
tout risque d'autant plus qu'on a vu que les années d'education(plus on
étudie mieux est le salaire) avait un effet positif, donc cela vont
opter pour une assurance santé,d'autant plus qu'on sait qu'aux états-uns
les assurances santé coûtent particuliérement chére.Et ces donc ce sont
ces individus là qu'arrive à capter la variables lnrevenu même s'ils ne
sont pas nombreux.

\hypertarget{discussion-et-limite}{%
\section{Discussion et Limite}\label{discussion-et-limite}}

Dans notre études ,je pense que l'environnement de l'étude est déjà un
peu affiné car ici on observe des individus bien précis de plus de 52
ans qui sont déjà dans des hopitaux et ayant des suivis.Parce que ce qui
nous frappe en premier c'est le fait que la variable statut\_santé ou
encore les variables associées à la santé ne soient aucunement
significatif alors que le but d'une assurance complémentaire c'est
justement d'alléger les coût liés à la santé.Pour moi d'un point de vue
personnel ça me semble un peu bizarre et je n'ai su trouver le pourquoi
et c'est peut-là la limite de l'étude.

\linebreak

Les variables mis en avance ne sont pour la plus part non liées à la
santé.On pourrait alors se dire que la santé n'a pas vraiment une
influence sur le fait qu'une personne opte ou pas pour une souscription
à une assurance complémentaire. Pour se faire,il faudrait aussi
peut-être faire une étude sur l'assurance santé auquelle les individus
ce sont souscrire et ce qu'elle couvre.

\end{document}
